


\subsection{Teorema de Plancherel}
Seja \(f\in L^2(\mathbb{T})\) e L=\(\pi\). Então:
\begin{itemize}

\item \(f= \lim_{N \to \infty} \displaystyle\sum_{-N}^{N} \hat{f}(n)e^{inx}\), na norma em \(L^2\);

\item \(\|f\|^2_{L^2}\)=\(\displaystyle\frac{1}{2\pi}\displaystyle\int_{\mathbb{T}}\|f(x)\|^2\quad dx\) = \( \displaystyle\sum_{-\infty}^{+\infty}|\hat{f}(n)|^2\);

\item \(\langle f,g\rangle_{L^2}\)=\(\displaystyle\frac{1}{2\pi}\displaystyle\int_{\mathbb{T}}f(x)\overline{g(x)}\quad dx\) = \(\displaystyle\sum_{-\infty}^{+\infty}\hat{f}(n)\overline{\hat{g}(n)}\);

\item Dada qualquer sucessão \(a_n\) tal que \(\displaystyle\sum_{-\infty}^{+\infty}|a_n|^2<\infty\), existe uma função \(f\in L^2(\mathbb{T})\) onde \(a_n=\hat{f}(n)\).
